% Chapter 1
% 
\chapter{Introduction} % Main chapter title
\label{chap:Chapter1} % For referencing the chapter elsewhere, use Chapter~\ref{Chapter1}


%-------------------------------------------------------------------------------
%---------
%
\section{Writing guidelines} 
\label{sec:chap1_guidelines} %For referencing this section elsewhere, use Section~\ref{sec:chap1_guidelines}

This chapter presents some recommendations to follow in the structuring and formatting of the dissertation or report.

The purpose of writing a dissertation or project report is, in essence, to document the work developed and highlight its importance. The author must follow the best writing rules, that is, respect grammatical and spelling rules, and correctly organize the text in order to make the message to be conveyed clear. 

The language used must be rigorous, scientific and without colloquial expressions. If in Portuguese, the document must be written using the rules of the new Orthographic Agreement that took effect in January 2009. If in English, the document should follow the rules of the United Kingdom variant. 

Automatic spelling correction must be complemented by consulting grammars and glossaries.

\section{Document structure}

The structure of the dissertation is normally divided into three main parts: Introduction, Body of the Thesis and Conclusions.

The Introduction contains, at least, a brief statement in accessible language of "what was done" (which is later described in the document); a brief summary of the survey of the state of the art in the domain or domains to which the thesis is dedicated; a clarification of the extent to which what has been done fits into this vision of the state of the art (problem statement and relevance) and how it contributes to its progress (goals and objectives); and a brief description of each of the following chapters.

The Body of the Thesis includes a survey of the state of the art carried out, normally placed in the chapter following the Introduction. The survey of the state of the art should not be too exhaustive, in order to give more space to the development of the chapters related to the work developed.

Afterwards follows a description, in writing, in successive chapters, of all the important points of the work carried out and the respective results, duly justified, from the proposed concepts and design to the (if existing) proof of concept implementation. A proper evaluation of the work should be provided, validating the proposed goals and objectives.

In the conclusions, a final balance of the work is made, highlighting the main aspects of "what was done", formulating critical judgments (positive and negative) about what was achieved, and launching suggestions for future work, if necessary. Note that, although the "conclusions" highlight the main aspects of what was done, as in the "introduction", the way in which this is done is completely different in each case: in the "introduction", the main aspects are presented to someone who has not yet read the thesis, so you should use more generic language; in the "conclusions", on the contrary, the language used is that which the thesis itself will have helped to build, and which the reader will now be able to understand.

\subsection{Contents checklist}

Table~\ref{tab:checklist} provides a list to check the contents of the dissertation. 

\begin{table}
\caption{Contents checklist.}
\label{tab:checklist}
\centering
\begin{tabular}{l l l}
\toprule
\tabhead{Description} & \tabhead{Mandatory?} & \tabhead{Page numbers} \\
\midrule
Cover (outside of template)&Yes	&None\\
First page (template first page)&Yes&None\\
Statement of Integrity&Yes&Roman\\
Portuguese abstract (Resumo)&Yes&Roman\\
English abstract &Yes&Roman\\
Acknowledgement&No&Roman\\
Table of Contents&Yes&Roman\\
List of Figures&Yes&Roman\\
List of Tables&Yes&Roman\\
Acronyms and Symbols&Yes&Roman\\
Dissertation Body (including Introduction and Conclusions)&Yes&Arabic\\
References&Yes&Arabic\\
Annexes and appendices&No&Arabic\\

\bottomrule\\
\end{tabular}
\end{table}


\section{Formatting}

The document can be written in Portuguese or English. The minimum number of pages is 60 and the maximum is 120. Complementary documentation may be added in the form of annexes, never exceeding 150 pages in total (including the main text of the thesis). 

Please follow the margins and fonts  defined in this template. The font size should be 11pt. The document should be printed double sided.

Note that the graphical aspect of the thesis is important, but does not replace a well-written and well-organised presentation of ideas.

Please refer to Chapter~\ref{chap:Chapter2} and Chapter~\ref{chap:Chapter3} for details about this template, how to format the document and insert citations, figures, tables, equations and other elements.

\section{Privacy and ethics}
The dissertation document should provide in the introduction a specific section with information on how privacy and ethics were considered in the analysis and development of the work. If not applicable, this should be justified.

\section{Use of Large Language Model tools}
Use of text generated from a large-scale language model (LLM), such as ChatGPT, is not allowed in the dissertation, unless the text itself is intended to be part of some experimental analysis.

Note that this does not preclude the use of LLM tools for editing or polishing text as long as the original text is produced by the author.

